\documentclass[12pt]{article}
\usepackage[utf8]{inputenc}
\usepackage[russian]{babel}
\usepackage{cmap}
\usepackage{pgfplots}
\usepackage{float}
\pgfplotsset{compat=1.7}% <-- moves axis labels near ticklabels (respects tick label widths)

\newcommand{\stdplot}[2]{
    \begin{tikzpicture}
    \begin{axis}[
        axis x line=middle,
        axis y line=left,
        xlabel=$x$, ylabel=$\mu 1$,
        width=16cm,
        height=8cm,
        cycle list={ }
    ]
    \addplot table {#1};
    \legend{#2}
    \end{axis}
    \end{tikzpicture}
}
\newcommand{\triplot}[4]{
    \begin{tikzpicture}
    \begin{axis}[
        axis x line=middle,
        axis y line=left,
        xlabel=$x$, ylabel=$\mu 1$,
        width=16cm,
        height=8cm,
        cycle list={ }
    ]
    \addplot table {#1};
    \addplot[densely dotted] table {#2};
    \addplot[dashed] table {#3};
    \legend{#4}
    \end{axis}
    \end{tikzpicture}
}

\usepackage{geometry} % пакет для установки полей
\geometry{top=3cm} % отступ сверху
\geometry{bottom=3cm} % отступ снизу
\geometry{left=2cm} % отступ справа
\geometry{right=2cm} % отступ слева

\begin{document}

\section{Постановка задачи}

\begin{equation}\label{eq:fmu}
f(x) = \mu_1
\end{equation}

\begin{enumerate}
    \item Реализовать программу, строящую график значений функции \ref{eq:fmu}, при этом высота окна фильтра равна высоте изображения, а ширина задается пользователем. Также шаг перемещения фильтра по изображению задается пользователем.
    \item Сделать анализ характерных точек графика.
    \item Реализовать возможность предобработки изображения, с использованием трех типов медианного фильтра. Размер фильтра задается пользователем.
\end{enumerate}

\begin{figure}[H]
    \center{
        \includegraphics[width=6cm]{1.png} %[width=8cm]
        \includegraphics[width=6cm]{2.png}
        \includegraphics[resolution=100]{3.png}
        % \triplot{./graphs/appTestSmall1}{./graphs/appTestMid1}{./graphs/appTestLarge1}{app=4, app=10, app=16}
    }
    \caption{Исходные изображения}
\end{figure}

\section{Выполнение}
Реализуем программу, отвечающую заданным требованиям. Исходный код программы приведен в приложении.
Рассмотрим график функции \ref{eq:fmu}. Для большей наглядности наложим график на исходное изображение.

\begin{figure}[H]
    \center{
        \includegraphics[resolution=100]{./images/appTestMid1_img.png} 
        \includegraphics[resolution=100]{./images/appTestSmall2_img.png} 
        \includegraphics[resolution=100]{./images/appTestSmall3_img.png} 
        \caption{Примеры графиков функции \ref{eq:fmu}}
    }
\end{figure}

\begin{figure}
    \center{
        \stdplot{./graphs/appTestMid1}{app=10 step=1}
    }
    \caption{Пример графика для 1 изображения}
\end{figure}

\begin{figure}
    \center{
        \stdplot{./graphs/appTestSmall2}{app=4 step=1}
    }
    \caption{Пример графика для 2 изображения}
\end{figure}

\begin{figure}
    \center{
        \stdplot{./graphs/appTestSmall3}{app=4 step=1}
    }
    \caption{Пример графика для 3 изображения}
\end{figure}

Как можно видеть из приведенных изображений и графиков, на границе двух символов функция имеет локальный минимум или локальный максимум.

\subsection{Рассмотрим влияние размера апертуры фильтра на поведение графика}

\begin{figure}
    \center{
        \triplot{./graphs/appTestSmall1}{./graphs/appTestMid1}{./graphs/appTestLarge1}{app=4, app=10, app=16}
    }
    \caption{Влияние размера апертуры фильтра для 1-го изображения}
    \label{img:app1}
\end{figure}

\begin{figure}
    \center{
        \triplot{./graphs/appTestSmall2}{./graphs/appTestMid2}{./graphs/appTestLarge2}{app=4, app=10, app=16}
    }
    \caption{Влияние размера апертуры фильтра для 2-го изображения}
    \label{img:app2}
\end{figure}

\begin{figure}
    \center{
        \triplot{./graphs/appTestSmall3}{./graphs/appTestMid3}{./graphs/appTestLarge3}{app=4, app=10, app=16}
    }
    \caption{Влияние размера апертуры фильтра для 3-го изображения}
    \label{img:app3}
\end{figure}

См рис \ref{img:app1}, \ref{img:app2} и \ref{img:app3}.

Как видно из графиков, для всех изображений с увеличением апертуры увеличиваются локальные максимумы и минимумы в абсолютных значениях, при этом сглаживаются небольшие шумы на графике.

\subsection{Рассмотрим влияние шага перемещения апертуры фильтра на графики}

\begin{figure}
    \center{
        \triplot{./graphs/stepTestSmall1}{./graphs/stepTestMid1}{./graphs/stepTestLarge1}{step=1, step=8, step=15}
    }
    \caption{Влияние шага перемещения апертуры для 1-го изображения}
    \label{img:step1}
\end{figure}

\begin{figure}
    \center{
        \triplot{./graphs/stepTestSmall2}{./graphs/stepTestMid2}{./graphs/stepTestLarge2}{step=1, step=8, step=15}
    }
    \caption{Влияние шага перемещения апертуры для 2-го изображения}
    \label{img:step2}
\end{figure}

\begin{figure}
    \center{
        \triplot{./graphs/stepTestSmall3}{./graphs/stepTestMid3}{./graphs/stepTestLarge3}{step=1, step=8, step=15}
    }
    \caption{Влияние шага перемещения апертуры для 3-го изображения}
    \label{img:step3}
\end{figure}

См рис \ref{img:step1}, \ref{img:step2} и \ref{img:step3}.

Как видно из графиков, при увеличении шага перемещения апертуры существенно снижается точность определения границ символов по графику, так как он становится более пологим и ступенчатым.

\subsection{Влияние предобработки изображения с использованием медианного фильтра на поведение графика}

\begin{figure}
    \center{
        \triplot{./graphs/medianCross1}{./graphs/medianPlus1}{./graphs/medianSquare1}{cross, plus, square}
    }
    \caption{Влияние формы медианного фильтра для 1-го изображения}
    \label{img:med1}
\end{figure}

\begin{figure}
    \center{
        \triplot{./graphs/medianCross2}{./graphs/medianPlus2}{./graphs/medianSquare2}{cross, plus, square}
    }
    \caption{Влияние формы медианного фильтра для 2-го изображения}
    \label{img:med2}
\end{figure}

\begin{figure}
    \center{
        \triplot{./graphs/medianCross3}{./graphs/medianPlus3}{./graphs/medianSquare3}{cross, plus, square}
    }
    \caption{Влияние формы медианного фильтра для 3-го изображения}
    \label{img:med3}
\end{figure}

См рис \ref{img:med1}, \ref{img:med2} и \ref{img:med3}.

Как видно из графиков применение медианного фильтра к изображениям, содержащим символы с заливкой их внутренних областей, почти не сказывается на поведении графика. Применение же фильтра к изображению, содержащему символы без внутренней заливки снижает шум на графике и способствует более четкому выделению максимумов и минимумов, находящихся на границе двух символов. 


\end{document}